%%%%%%%%%%%%%%%%%%%%%%%%%%%%%%%%%%%%%%%%%%%%%%%%%%%%%%%%%%%%%%%%%%%%%%%%%%%%%%%% 
% Thin Sectioned Essay                                                         %
% LaTeX Template                                                               %
% Version 1.0 (3/8/13)                                                         %
%                                                                              % 
% This template has been downloaded from:                                      %
% http://www.LaTeXTemplates.com                                                %
%                                                                              % 
% Original Author:                                                             %
% Nicolas Diaz (nsdiaz@uc.cl) with extensive modifications by:                 %
% Vel (vel@latextemplates.com)                                                 %
%                                                                              % 
% License:                                                                     %
% CC BY-NC-SA 3.0 (http://creativecommons.org/licenses/by-nc-sa/3.0/)          %
%                                                                              % 
%%%%%%%%%%%%%%%%%%%%%%%%%%%%%%%%%%%%%%%%%%%%%%%%%%%%%%%%%%%%%%%%%%%%%%%%%%%%%%%% 

% Font size (can be 10pt, 11pt or 12pt) and paper size (remove a4paper for US
% letter paper)
\documentclass[a4paper, 11pt]{article}
\usepackage[margin=0.5in]{geometry}

\usepackage[protrusion=true,expansion=true]{microtype} % Better typography
\usepackage[utf8]{inputenc} % Spanish characters
\usepackage[T1]{fontenc}    % Required for accented characters
\usepackage{graphicx}       % Required for including pictures
\usepackage{wrapfig}        % Allows in-line images
\usepackage{hyperref}       % Allows the use of hyperlinks
\usepackage{mathpazo}       % Use the Palatino font
\usepackage{amsmath}        % Allows align
\usepackage{listings}      	% Allows code 
\usepackage{float}         
\usepackage{subfig}
\usepackage{amsmath}
\usepackage{multirow}
\usepackage{multicol}
\usepackage{rotating}
\usepackage{makecell}
\usepackage{diagbox}

% ------------------------------------------------------------------------------

\lstset{basicstyle=\footnotesize\ttfamily, breaklines=true}

\graphicspath{{../plots/}}

\captionsetup[table]{name=Tabla}

% Change line spacing here, Palatino benefits from a slight increase by default
\linespread{1.05}

\makeatletter

% ------------------------------------------------------------------------------

% Change the square brackets for each bibliography item from '[1]' to '1.'
\renewcommand\@biblabel[1]{\textbf{#1.}}

% Reduce the space between items in the itemize and enumerate environments and
% the bibliography
\renewcommand{\@listI}{\itemsep=0pt}

% Customize the title - do not edit title and author name here, see the TITLE
% block below
\renewcommand{\maketitle}{
  \begin{flushright} % Right align
    % Increase the font size of the title
    {\LARGE\@title}
    % Some vertical space between the title and author name
    \vspace{50pt}

    % Author name
    {\large\@author}\\
    % Date
    \@date
    % Some vertical space between the author block and abstract
    \vspace{40pt}
  \end{flushright}
}

% ------------------------------------------------------------------------------
%	TITLE
% ------------------------------------------------------------------------------

\title{\textbf{Trabajo Práctico Final\\ Análisis completo de un conjunto de
    datos}}

\author{
  \textsc{Agustín Mista}\\
  \textit{Universidad Nacional de Rosario}\\
  \textit{Tópicos de Minería de Datos}
}

\date{Rosario, 28 de Diciembre de 2017}

% ------------------------------------------------------------------------------

\begin{document}

\maketitle % Print the title section

\pagebreak

% ------------------------------------------------------------------------------
%	ESSAY BODY
% ------------------------------------------------------------------------------

\section*{Origen de los datos}

Para este trabajo final utilizamos el dataset \textit{Spam
  Base}\footnote{\url{https://archive.ics.uci.edu/ml/datasets/spambase}},
orientado a la clasificación de emails en deseados (\textit{Ham}) y no-deseados
(\textit{Spam}). El mismo fue creado utilizando una base de datos de emails
reales de la compañia Hewlett-Packard en 1998 y estudiado en el artículo
Spam!\footnote{Cranor, Lorrie F., LaMacchia, Brian A. Spam! Communications of
  the ACM, 41(8):74-83, 1998.}.

El concepto de Spam es diverso: publicidades de productos/páginas web, esquemas
piramidales, cadenas de correo, pornografía, etc. La prevención (mediante
filtrado) de esta clase de emails es una tarea que involucra reconocer ciertos
patrones en el contenido de los mails entrantes. Debido a ésto, resulta
interesante poseer de un conjunto previamente clasificado de correo entrante del
que podamos extraer estos patrones con el fin de crear filtros de Spam
personalizados, filtrando efectivamente todo el correo no deseado, con el mínimo
número posible de falsos positivos.

Este dataset cuenta con 4601 muestras, cada una con 57 features que describen la
frecuencia de aparición de ciertas palabras (prefijo \texttt{word}) o caracteres
(prefijo \texttt{char}) clave obtenidos a partir de los emails analizados, junto
con las frecuencias de aparición de secuencias de caracterese capitalizados
(prefijo \texttt{capital}).

A continuación se muestra la lista de features de este dataset, ordenadas según
el índice de su respectiva columna en los datos.

\begin{multicols}{4}
\begin{enumerate}
\item \texttt{word\_make}         
\item \texttt{word\_address}      
\item \texttt{word\_all}          
\item \texttt{word\_3d}           
\item \texttt{word\_our}          
\item \texttt{word\_over}         
\item \texttt{word\_remove}       
\item \texttt{word\_internet}     
\item \texttt{word\_order}        
\item \texttt{word\_mail}         
\item \texttt{word\_receive}      
\item \texttt{word\_will}         
\item \texttt{word\_people}       
\item \texttt{word\_report}       
\item \texttt{word\_addresses}    
\item \texttt{word\_free}         
\item \texttt{word\_business}     
\item \texttt{word\_email}        
\item \texttt{word\_you}          
\item \texttt{word\_credit}       
\item \texttt{word\_your}         
\item \texttt{word\_font}         
\item \texttt{word\_000}          
\item \texttt{word\_money}        
\item \texttt{word\_hp}           
\item \texttt{word\_hpl}          
\item \texttt{word\_george}       
\item \texttt{word\_650}          
\item \texttt{word\_lab}          
\item \texttt{word\_labs}         
\item \texttt{word\_telnet}       
\item \texttt{word\_857}          
\item \texttt{word\_data}         
\item \texttt{word\_415}          
\item \texttt{word\_85}           
\item \texttt{word\_technology}   
\item \texttt{word\_1999}         
\item \texttt{word\_parts}        
\item \texttt{word\_pm}           
\item \texttt{word\_direct}       
\item \texttt{word\_cs}           
\item \texttt{word\_meeting}      
\item \texttt{word\_original}     
\item \texttt{word\_project}      
\item \texttt{word\_re}           
\item \texttt{word\_edu}          
\item \texttt{word\_table}        
\item \texttt{word\_conference}   
\item \texttt{char\_;}           
\item \texttt{char\_(}            
\item \texttt{char\_[}           
\item \texttt{char\_!}            
\item \texttt{char\_\$}            
\item \texttt{char\_\#}     
\item \texttt{capital\_average} 
\item \texttt{capital\_longest} 
\item \texttt{capital\_total}   
\end{enumerate}
\end{multicols}

\paragraph{Preprocesamiento}

En las siguientes secciones se muestran resultados obtenidos a partir tanto del
dataset original, como de alguna combinación de preprocesamientos logarítmico
(\texttt{log()}), escalado (\texttt{scale()}) o Principal Component Analysis
(\texttt{prcomp()}), los cuales se detallan en cada caso particuar.

% ------------------------------------------------------------------------------

\section*{Visualización de los datos}

Ya que este dataset cuenta con un gran número de features, la manera más
intuitiva de visualizar los datos en un gráfico de menor dimensionalidad es
efectuar una PCA sobre los mismos. A continuación se muestra este análisis sobre
nuestros datos, utilizando colores para separar las clases de emails y etiquetas
para señalar hacia dónde y con qué relevancia aporta evidencia cada feature.

\begin{figure}[H]
  \captionsetup[subfigure]{justification=centering, labelformat=empty}
  \makebox[\textwidth][c]{
    \subfloat[][Principal Component Analysis]
    {\includegraphics[width=\textwidth]{{spam.pca.loadings}.png}}
  }
\end{figure}

A primera vista, podemos notar la existencia de dos ejes sobre los cuales se
situan las features más relevantes de nuestro dataset, y que se corresponden con
aquellos emails deaseados o basura cada caso. Ambos ejes se se unen en el origen
de nuestro gráfico, formando una nube densa de muestras donde no existe
suficiente evidencia por parte de las features de menor relevancia como para
lograr separar cada muestra según su clase de manera efectiva. Suponemos que la
mayoría de los errores de clasificación deberían producirse en esa zona.

Si ampliamos el eje correspondiente al correo deseado, podemos observar como la
presencia de palabras como ``telnet'', ``lab'', ``hp'' o códigos de área
telefónica son buenos indicadores de que estamos ante un correo deseado. Resulta
evindente que estas palabras están relacionadas directamente con los intereses
de la empresa de donde provienen los emails analizados, por lo que debería
usarse un dataset de origen más amplio si se desea construir un filtro de Spam
efectivo en entornos más diversos.

\begin{figure}[H]
  \captionsetup[subfigure]{justification=centering, labelformat=empty}
  \makebox[\textwidth][c]{
    \subfloat[][]{\includegraphics[width=\textwidth]
      {{spam.pca.loadings.ham}.png}}}
\end{figure}

Si en cambio ampliamos el eje correspondiente al correo no deseado, podemos
observar como la presencia de palabras como ``money'', ``free'', ``your'',
caracteres como ``\$'', ``!'' o largas secuencias de caracteres capitalizados
son buenos indicadores de correo basura. 

\begin{figure}[H]
  \captionsetup[subfigure]{justification=centering, labelformat=empty}
  \makebox[\textwidth][c]{
    \subfloat[][]{\includegraphics[width=\textwidth]
      {{spam.pca.loadings.spam}.png}}}
\end{figure}

% ------------------------------------------------------------------------------

\section*{Análisis de features relevantes}

Lorem ipsum dolor sit amet, consectetuer adipiscing elit. Donec hendrerit tempor
tellus. Donec pretium posuere tellus. Proin quam nisl, tincidunt et, mattis
eget, convallis nec, purus. Cum sociis natoque penatibus et magnis dis
parturient montes, nascetur ridiculus mus. Nulla posuere. Donec vitae dolor.
Nullam tristique diam non turpis. Cras placerat accumsan nulla. Nullam rutrum.
Nam vestibulum accumsan nisl.

\vspace{20pt}
\begin{center}
\begin{tabular}{c  c | c | c | c | c | c | c | c | c | c | c}  
  \multirowcell{3}{Forward\\ Selection}
  & RF  & 53 & 7  & 52 & 57 & 19 & 50 & 16 & 56 & 25 & 46 \\ 
  & LDA & 53 & 52 & 56 & 7  & 25 & 8  & 5  & 21 & 46 & 42 \\ 
  & SVM & 53 & 7  & 52 & 25 & 57 & 46 & 16 & 21 & 5  & 8  \\
  \hline
  \multirowcell{3}{Backward\\ Elimination}
  & RF  & 53 & 7  & 52 & 56 & 25 & 5  & 46 & 27 & 19 & 16 \\ 
  & LDA & 52 & 57 & 7  & 25 & 24 & 23 & 16 & 21 & 46 & 8  \\
  & SVM & 53 & 7  & 52 & 25 & 57 & 46 & 5  & 27 & 42 & 16 \\ 
  \hline
  \multirowcell{2}{Recursive Feature\\ Elimination}
  & RF  & 7  & 25 & 53 & 52 & 55 & 16 & 46 & 21 & 27 & 5  \\
  & SVM & 27 & 41 & 25 & 46 & 26 & 42 & 57 & 53 & 7  & 16 \\ 
  \hline
  Kruskal-Wallis & & 52 & 53 & 7  & 56 & 16 & 21 & 55 & 24 & 57 & 23 \\ 
\end{tabular}
\end{center}
\vspace{20pt}

% ------------------------------------------------------------------------------

\section*{Clustering}

Pellentesque dapibus suscipit ligula. Donec posuere augue in quam. Etiam vel
tortor sodales tellus ultricies commodo. Suspendisse potenti. Aenean in sem ac
leo mollis blandit. Donec neque quam, dignissim in, mollis nec, sagittis eu,
wisi. Phasellus lacus. Etiam laoreet quam sed arcu. Phasellus at dui in ligula
mollis ultricies. Integer placerat tristique nisl. Praesent augue. Fusce
commodo. Vestibulum convallis, lorem a tempus semper, dui dui euismod elit,
vitae placerat urna tortor vitae lacus. Nullam libero mauris, consequat quis,
varius et, dictum id, arcu. Mauris mollis tincidunt felis. Aliquam feugiat
tellus ut neque. Nulla facilisis, risus a rhoncus fermentum, tellus tellus
lacinia purus, et dictum nunc justo sit amet elit.

\begin{figure}[H]
  \captionsetup[subfigure]{justification=centering, labelformat=empty}
  \makebox[\textwidth][c]{
    \subfloat[][]{\includegraphics[width=\textwidth]
      {{spam.km.o}.png}}}
\end{figure}

\begin{figure}[H]
  \captionsetup[subfigure]{justification=centering, labelformat=empty}
  \makebox[\textwidth][c]{
    \subfloat[][]{\includegraphics[width=\textwidth]
      {{spam.km.lsp}.png}}}
\end{figure}

\begin{figure}[H]
  \captionsetup[subfigure]{justification=centering, labelformat=empty}
  \makebox[\textwidth][c]{
    \subfloat[][]{\includegraphics[width=\textwidth]
      {{spam.km.l}.png}}}
\end{figure}

% ------------------------------------------------------------------------------

\section*{Clasificación}

Nullam eu ante vel est convallis dignissim. Fusce suscipit, wisi nec facilisis
facilisis, est dui fermentum leo, quis tempor ligula erat quis odio. Nunc porta
vulputate tellus. Nunc rutrum turpis sed pede. Sed bibendum. Aliquam posuere.
Nunc aliquet, augue nec adipiscing interdum, lacus tellus malesuada massa, quis
varius mi purus non odio. Pellentesque condimentum, magna ut suscipit hendrerit,
ipsum augue ornare nulla, non luctus diam neque sit amet urna. Curabitur
vulputate vestibulum lorem. Fusce sagittis, libero non molestie mollis, magna
orci ultrices dolor, at vulputate neque nulla lacinia eros. Sed id ligula quis
est convallis tempor. Curabitur lacinia pulvinar nibh. Nam a sapien.

\begin{figure}[H]
  \captionsetup[subfigure]{justification=centering, labelformat=empty}
  \makebox[\textwidth][c]{
    \subfloat[][]{\includegraphics[width=0.5\textwidth]
      {{spam.boosting.error}.png}}}
\end{figure}

\begin{figure}[H]
  \captionsetup[subfigure]{justification=centering, labelformat=empty}
  \subfloat[][]{\includegraphics[width=0.5\textwidth]
    {{spam.svm.poly.error}.png}}
  \captionsetup[subfigure]{justification=centering, labelformat=empty}
  \subfloat[][]{\includegraphics[width=0.5\textwidth]
    {{spam.svm.rbf.error}.png}}
\end{figure}

% \begin{figure}[H]
% \captionsetup[subfigure]{justification=centering, labelformat=empty}
%   \centering
%   \subfloat[][SVM con kernel \texttt{polynomial}]
%   {\includegraphics[width=0.5\textwidth]{{lampone.svm.poly.error}.png}}
%   \subfloat[][SVM con kernel \texttt{radial}]
%   {\includegraphics[width=0.5\textwidth]{{lampone.svm.rbf.error}.png}}
% \end{figure}

% ------------------------------------------------------------------------------

\end{document}
